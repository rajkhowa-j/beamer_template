%!TEX program = xelatex
\documentclass[10pt]{beamer}
\usepackage{amsmath,amsthm}
\usepackage{amssymb}
\usepackage{graphics,graphicx}
%\usepackage{amsrefs}
\usepackage{ragged2e}
\usepackage{pst-all}
\usepackage{pstricks}
\usepackage{csquotes}

\usepackage[%
  backend=biber,
  style=nature,
  citestyle=verbose-ibid,
  sorting=nty,
  natbib=true,
]{biblatex}
\addbibresource{ref.bib}
\AtBeginBibliography{\small}
\DeclareNameAlias{default}{family-given}
%\renewcommand{\scriptsize}{\tiny}
\renewcommand{\footnotesize}{\tiny}
\def\B(G){\mathcal{T}_{\mathcal{B}_G}}
\def\bpsp{\begin{pspicture}}
	\def\epsp{\end{pspicture}}
\newcommand\arrow[2]{\rotatebox{#1}{\scalebox{2}{\psline[linecolor=#2]{->}(0,0)(.1,0)}}}
\renewcommand{\baselinestretch}{1.3}
\def\w{{\sf w}}
\usetheme{Boadilla}
\usefonttheme{serif}
\usefonttheme{professionalfonts}
\mode<presentation>
\setbeamertemplate{frame numbering}[fraction]
\setbeamertemplate{footline}[frame number]
\setbeamertemplate{caption}[numbered]

\title{Progress Seminar: Autumn 2024\\ Area of research: }
\author{ \\by\\ Name (Roll Number)\\Under the supervision of Dr. Name of the supervisor}
\vspace{.3cm}
\institute{\includegraphics[scale=0.32]{tulogo.png}\\Department of Mathematical Sciences\\ Tezpur University}\vspace{-.2mm}
\date{19.12.2024}

\setbeamertemplate{navigation symbols}{}

\begin{document}
\nocite{*}
\frame[plain]{\titlepage}

\begin{frame}
\frametitle{Outline}
\hfill
\parbox[t]{.90\textwidth}{
  \begin{minipage}[c][0.55\textheight]{\textwidth}
  \tableofcontents
  \end{minipage}
  }
\end{frame}
\section{Introduction}
    \begin{frame}\frametitle{Introduction}
    \small
	\begin{itemize}
		 \justifying
		\item A \textbf{graph} $G$ consists of a finite nonempty set $V$ of objects called vertices and a set $E$ of $2$-element subsets of $V$ called edges. The sets $V$ and $E$ are the vertex set and edge set of $G$, respectively.
        
	\end{itemize}
		\begin{figure}
            \centering
			\begin{pspicture}(1,-0.5)(6,1)
				\psdots[](2,0)(3,0)(4,0)(5,0)(3,1)(4,1)
				\psline[](2,0)(3,0)(4,0)(5,0)
                    \psline[](3,0)(3,1)(4,1)(4,0)
			\end{pspicture}
			\caption{A graph on six vertices and six edges.}
			\label{fig:graph}
		\end{figure}
\end{frame}
\begin{frame}{Introduction}
\small
    \begin{itemize}\justifying
        \item There are several matrices associated with a graph, namely, the Adjacency matrix, Incidence matrix, Laplacian matrix and Distance matrix.
        
        
        \item Let $G$ be a connected graph on $n$ vertices with $V(G)=\{1, 2,\ldots,n\}$. Then, the \textit{adjacency matrix} $A(G)=[a_{ij}]$ of $G$ is defined to be the $n\times n$ matrix with $$a_{ij}=\begin{cases}
            1 & \text{if}\,\,\, i\sim j\\
            0 & \text{otherwise}.
        \end{cases}$$
    \end{itemize}
\end{frame}
\begin{frame}\frametitle{Irreducible Matrix}
\small
    \justifying
    \begin{itemize}
        \item A nonnegative square matrix of order $n$, $n\geq 2$, is called reducible if there exists a permutation matrix $P$ such that $$X=P^T\begin{pmatrix}
			B & C \\
			0 & D
		\end{pmatrix}P,$$ where $B$ and $D$ are square submatrices. A nonnegative square matrix of order $n$, $n\geq2$, is called \textbf{irreducible} if it is not reducible.
    \end{itemize}
        
		
		\begin{example}
			The given matrix $$A=\begin{pmatrix}
			    0 & 1 & 0 & 0 \\
                    1 & 0 & 1 & 1 \\
                    0 & 1 & 0 & 0 \\
                    0 & 1 & 0 & 0 \\
			\end{pmatrix}$$
            is irreducible.
		\end{example}
\footnotetext{\cite{minc1988nonnegative}}
\end{frame}
\begin{frame}{Properties of Irreducible matrix}
\small
    \justifying
    \begin{itemize}
        \item If $A$ is a $n\times n$ irreducible matrix then $(I_n+A)^{n-1}>0$.
        
        \item If $A=[a_{ij}]$ is an irreducible matrix then for each $(i,j)$ there exists an integer $k$ such that $a_{ij}^{(k)}>0$.
        
        \item The spectral radius of an irreducible matrix $A$, $\rho(A)$ is a simple eigenvalue of it. Also there is a positive eigenvector corresponding to $\rho(A)$.
        
        \item An irreducible matrix has exactly one eigenvector in the set $$\mathbb{E}^n=\left\{(x_1,x_2,\ldots,x_n)\in\mathbb{P}^n\bigg|\sum_{i=1}^{n}{x_i}=1\right\},$$where $\mathbb{P}$ denote the set of all nonnegative real numbers. 
\end{itemize}
\footnotetext{\cite{minc1988nonnegative}}
\end{frame}
\section{Results}
\begin{frame}{}
\small
    \begin{example}
			\begin{figure}\centering
				\begin{pspicture}(-3,-1)(3,1)
                        %\psgrid[subgriddiv=1,griddots=10,gridlabels=7pt](-3,-1)(3,1)
                        \psdots[](-1.5,0)(0,0)(1.5,0)(-0.707,0.707)(0.707,0.707)(0,-1)(2.5,0)
                        \psline[](-1.5,0)(0,0)(1.5,0)(2.5,0)
                        \psline[](0,0)(0,-1)
                        \psline[](-0.707,0.707)(0,0)(0.707,0.707)
				\end{pspicture}
				\caption{T}
				\label{fig:path}
			\end{figure}
		\end{example}
\end{frame}
\section{Future Plans}
\begin{frame}{Future Plans}

\end{frame}
\begin{frame}[allowframebreaks]\frametitle{References}
    \printbibliography[heading=bibnumbered]
\end{frame}
\begin{frame}
	\begin{center}
		{\fontsize{30}{40}\selectfont\textcolor{blue}{Thank You!}}
	\end{center}
\end{frame}
\end{document}